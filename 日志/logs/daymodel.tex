\section{2024年4月27日} % 日期作为章节显示在目录中

\subsection{今日进展} % 今日进展
今天完成了五子棋的基本棋盘绘制,并实现了棋子的下放功能。

\subsection{成果展示} % 成果展示

\subsubsection{程序界面截图}
\begin{figure}[h]
    \centering
    % \includegraphics[width=0.7\textwidth]{images/screenshot_0427.png} % 插入图片,确保文件存在
    % 如果没有图片,可以使用以下占位图
    \fbox{\parbox[b][5cm][c]{0.7\textwidth}{\centering 程序界面截图(2024年4月27日)占位图}}
    \caption{程序界面截图(2024年4月27日)}
    \label{fig:screenshot_0427}
\end{figure}

\subsubsection{运行结果}
程序成功运行,能够显示棋盘并响应玩家的点击操作。

\subsection{问题与解决方案} % 问题与解决方案

\subsubsection{遇到的问题}
在实现棋子下放功能时,点击事件无法正确获取鼠标位置,导致棋子无法准确放置。

\subsubsection{解决方法}
通过调整事件处理函数,使用相对坐标系计算鼠标点击位置,并将其映射到棋盘格子上,实现了准确放置棋子的功能。

\subsection{代码分析} % 代码分析

\subsubsection{核心算法}
\begin{lstlisting}[caption={五子棋算法核心代码(2024年4月27日)}, label={code:core_0427}]
def check_win(board, player):
    # 检查横向是否连成五子
    for i in range(len(board)):
        for j in range(len(board[i]) - 4):
            if all(board[i][j+k] == player for k in range(5)):
                return True
    # 检查纵向是否连成五子
    for i in range(len(board) - 4):
        for j in range(len(board[i])):
            if all(board[i+k][j] == player for k in range(5)):
                return True
    # 检查斜向(左上到右下)是否连成五子
    for i in range(len(board) - 4):
        for j in range(len(board[i]) - 4):
            if all(board[i+k][j+k] == player for k in range(5)):
                return True
    # 检查斜向(右上到左下)是否连成五子
    for i in range(len(board) - 4):
        for j in range(4, len(board[i])):
            if all(board[i+k][j-k] == player for k in range(5)):
                return True
    return False
\end{lstlisting}

\subsubsection{代码解释}
上述代码实现了五子棋的胜利条件检测,包括横向、纵向以及两种斜向的五子连线检查。

\subsection{明日计划} % 明日计划
计划实现AI对战功能,优化胜利条件检测算法,并进行界面美化。

\subsection{心得体会} % 心得体会
通过今天的开发,深入理解了事件处理和坐标转换的实现方法,对Python的GUI编程有了更深入的认识。
