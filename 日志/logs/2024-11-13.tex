\section{2024年11月13日} % 日期作为章节显示在目录中

\subsection{今日进展} % 今日进展
今天了解了作业的具体要求,了解了使用C++实现五子棋对弈程序所需要的算法基础,即Min-Max算法和Alpha-Bata剪枝优化。
然后在GitHub上创建了仓库,方便后续的版本控制和更新。并使用GPT生成了latex模板方便后续开发日志的记录。
同时今天找到了一些宝贵的学习参考资源,比如GitHub上基于Javascript语言的五子棋AI教程\url{https://github.com/lihongxun945/gobang?tab=readme-ov-file}
和bilibili上的算法教程视频\url{https://www.bilibili.com/video/BV1v94y1r7F8/?spm_id_from=333.880.my_history.page.click&vd_source=2f0075ad419feeef529bb2dce0adc975}。


\subsection{本次作业的基本要求} 

\subsubsection{五子棋详细规则}

黑白双方轮流落子,黑方为先手。

在横、竖、斜方向上连成五子(连续五个棋子皆为己方)者为胜。

黑棋在行棋过程中,如果违反以下“禁手规则”会被判负。

三三禁手:黑棋在一个位置下子后,形成两个或两个以上的活三。活三是指在棋盘上有三个连续的黑子,并且两端都有空位可以继续下子形成五连珠。

四四禁手:黑棋在一个位置下子后,形成两个或两个以上的活四。活四是指在棋盘上有四个连续的黑子,并且至少有一个空位可以继续下子形成五连珠。

长连禁手:黑棋在一个位置下子后,形成六个或更多连续的黑子。

四三禁手:黑棋在一个位置下子后,同时形成一个活四和一个活三。这种情况也被视为禁手。

注意到这里的禁手规则,后续需要特定的函数实现。

\subsubsection{其他要求}
棋盘大小可以自定义,如果要参加Botzone \url{https://botzone.org.cn/} 比赛,则棋盘大小为15*15。

\subsection{算法基础介绍}

Min-Max算法:

五子棋看起来有各种各样的走法,而实际上把每一步的走法展开,就是一颗巨大的博弈树。在这个树中,从根节点为0开始,奇数层表示电脑可能的走法,偶数层表示玩家可能的走法。

那么我们如何才能知道哪一个分支的走法是最优的,我们就需要一个评估函数能对当前整个局势作出评估,返回一个分数。我们规定对电脑越有利,分数越大,对玩家越有利,分数越小,分数的起点是0。

我们遍历这颗博弈树的时候就很明显知道该如何选择分支了:

电脑走棋的层我们称为 MAX层,这一层电脑要保证自己利益最大化,那么就需要选分最高的节点。

玩家走棋的层我们称为MIN层,这一层玩家要保证自己的利益最大化,那么就会选分最低的节点。

而每一个节点的分数,都是由子节点决定的,因此我们对博弈树只能进行深度优先搜索而无法进行广度优先搜索。深度优先搜索用递归非常容易实现,然后主要工作其实是完成一个评估函数,这个函数需要对当前局势给出一个比较准确的评分。

alpha-beta剪枝:
即每次更新节点的数值时,查看其是否被父节点所兼容:如果父节点已经得到了合理的结果,就可以通过break语句进行剪枝。


\subsection{明日计划} % 明日计划
编写一些基本的函数,实现输入与输出的读取。

\subsection{后续待实现的内容} % 心得体会

胜负判断函数;禁手规则判断函数;局势评估函数;
